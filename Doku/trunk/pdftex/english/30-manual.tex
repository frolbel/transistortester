\chapter{Instructions for use}
\label{sec:manual}
\section{The measurement operation}
Using of the Transistor-Tester is simple.
Anyway some hints are required.
In most cases are wires with alligator clips connected to the test ports with plugs.
Also sockets for transistors can be connected.
In either case you can connect parts with three pins to the three test ports in any order.
If your part has only two pins, you can connect this pins to any two of the tree test ports.
Normally the polarity of part is irrelevant, you can also connect pins of electrolytical capacitors in any order. 
The measurement of capacity is normally done in a way, that the minus pole is at the test port with the lower number.
But, because the measurment voltage is only between \(0.3V\) and at most \(1.3V\), the polarity doesn\'t matter.
When the part is connected, you should not touch it during the measurement. You should put it down to a nonconducting pad
if it is not placed in a socket. You should also not touch to the isolation of wires connected with the test ports.
Otherwise the measurement results can be affected.
Then you should press the start button.
After displaying a start message, the measurement result should appear after two seconds.
If capacitors are measured, the time to result can be longer corresponding to the capacity.

How the transistor-tester continues, depends on the configuration of the software.
\begin{description}
  \item[Single measurement mode] If the tester is configured for single measurement mode (POWER\_OFF option), the tester shut off automatical 
after displaying the result for 28~seconds for a longer lifetime of battery. 
During the display time a next measurement can be started by pressing the start button.
After the shut off a next measurement can be started too of course.
The next measurement can be done with the same or another part.
If you have not installed the electronic for automatic shut down, your
last measurement result will be displayed until you start the next measurement.

  \item[Endless measurement mode] A special case is the configuration without automatical shut off.
For this case the POWER\_OFF option is not set in the Makefile.
This configuration is normally only used without the transistors for the shut off function.
A external off switch is necessary for this case. The tester will repeat measurements until power
is switched off.

  \item[Multi measurement mode] In this mode the tester will shut down not after the first measurement but 
after a configurable series of measurements.
For this condition a number (e.g.~5) is assigned to the POWER\_OFF option.
In the standard case the tester will shut down after five
measurements without found part. If any part is identified by test, the tester is shut down after double of
five (ten) measurements. A single measurement with unknown part after a series of measurement of known parts will
reset the counter of known measuerements to zero. Also a single measurement of known part will reset the counter
of unknown measurements to zero. This behavior can result in a nearly endless series of measurements without
pressing the start button, if parts are disconnected and connected in periodical manner.

In this mode there is a special feature for the display period. If the start button is pressed only short for switching
on the tester, the result of measurement ist only shown for 5 seconds. Buf if you press and hold the start button until
the first message is shown, the further measurement results are shown for 28~seconds.
The next measurement can started earlier by pressing the start button during the displaying of result.

\end{description}

\section{Optional menu functions for the ATmega328}
If the menu function is selected, the tester start a selection menu after a long key press (\textgreater 500ms)
for additional functions.
This function is also available for other processors with at least 32K flash memory.
The selectable functions are shown in row two of a 2-line display or as marked function in row 3 of a 4-line display.
The previous and next function is also shown in row 2 and 4 of the display in this case.
After a longer wait time without any interaction the program leave the menu and returns to the normal transistor tester function.
With a short key press the next selection can be shown.
A longer key press starts the shown or marked function.
After showing the last function ''switch off'', the first function will be shown next.

If your tester has also the rotary pulse encoder installed, you can call the menu with the additional functions
also with a fast rotation of the encoder during the result of a previous test is shown.
The menu functions can be selected with slow rotation of the encoder in every direction.
Starting of the selected menu function can only be done with a key press.
Within a selected function parameters can be selected with slow rotation of the encoder.
A fast rotarion of the encoder will return to the selection menu.

\begin{description}
 \item[frequency]
The additional function ''frequency'' (frequency measurement) uses the ATmega Pin PD4, which is also connected to the LCD.
First the frequency is allways measured by counting.
If the measured frequency is below \(25kHz\), additionally the mean period of the input signal
is measured and with this value the frequency is computed with a resolution of up to \(0.001Hz\).
By selecting the POWER\_OFF option in the Makefile, the period of frequency measurement is limited to 8~minutes.
The frequency measurement will be finished with a key press and the selectable functions are shown again.\\

 \item[f-Generator]
With the additional function ''f-Generator'' (frequency generator) the selectable frequencies can be switched with key presses.
After selecting the last choise of frequencies, the generator is switched back to
the first frequency next (cyclical choise).
If the POWER\_OFF option is selected in the Makefile, the key must be pressed longer, because a
short key press (\textless~0.2s) only reset the time limit of 4~minutes.
The elapsed time is shown with a point for every 30 seconds in row 1 of the display.
With periodical short key press you can prevent the time out of the frequency generation.
With a long key press (\textgreater~0.8s) you will stop the frequency generator and return to the function menu.\\

\item[10-bit PWM]
The additional function ''10-bit PWM'' (Pulse Width Modulation) generates a fixed frequency with selectable
puls width at the pin TP2.
With a short key press (\textless~0.5s) the pulse width is increased by \(1\%\), with a longer key press the pulse width
is increased by \(10\%\).
If \(99\%\) is overstepped, \(100\%\) is subtracted from the result.
If the POWER\_OFF option is selected in the Makefile, the frequency generation is finished after 8~minutes without any key press.
The frequency generation can also be finished with a very long key press (\textgreater~1.3s).\\

\item[C+ESR@TP1:3]
The additional function ''C+ESR@TP1:3'' selects a stand-alone capacity measurement with 
ESR (Equivalent Series Resistance) measurement at the test pins TP1 and TP3. 
Capacities from \(2\mu F\) up to \(50mF\) can be measured. 
Because the measurement voltage is only about \(300mV\), in most cases the capacitor can be
measured ''in circuit'' without previous disassembling.
If the POWER\_OFF option is selected in the Makefile, the count of measurements is limited
to 250, but can be started immediately again.
The series of measurements can be finished with a long key press.\\

 \item[Resistor meter]
With the \mbox{1 \electricR 3} symbol the tester changes to a resistor meter at TP1 and TP3 . 
This operation mode will be marked with a {\bf[R]} at the right side of the first display line.
Because the ESR measurement is not used in this operation mode, the resolution of the measurement
result for resistors below \(10\Omega\) is only \(0.1\Omega\).
If the resistor measurement function is configured with the additional inductance measurement,
a \mbox{1 \electricR \electricL 3} symbol is shown at this menu.
Then the resistor meter function includes the measurement of inductance for resistors below \(2100\Omega\).
At the right side of the first display line a {\bf[RL]} is shown.
For resistors below \(10\Omega\) the ESR measurement is used,
if no inductance is find out. For this reason the resolution for resistors below \(10\Omega\) 
is increased to \(0.01\Omega\).
With this operation mode the measurement is repeated without any key press.
With a key press the tester finish this operation mode and returns to the menu.
The same resistor meter function is started automatically, if a single resistor is connected between TP1 and TP3
and the start key was pressed in the normal tester function. In this case the tester returns
from the special mode opration to the normal tester function with a key press.

 \item[Capacitor meter]
With the \mbox{1 \electricC 3} symbol the tester changes to a capacitor meter function at TP1 and TP3.
This operation mode will be marked with a {\bf[C]} at the right side of the first display line.
With this operation mode capacitors from \(1pF\) up to \(100mF\) can be measured.
In this operation mode the measurement is repeated without key press.
With a key press the tester finish this operation mode and returns to the menu.
In the same way as with resistors, the tester changes automatically to the capacitor meter function,
if a capacitor between TP1 and TP3 is measured with the normal tester function.
After a automatically start of the capacitor meter function the tester returns with a key press to
the normal tester function.

\item[rotary encoder]
With the function ''rotary encoder'' a rotary encoder can be checked.
The three pins of the rotary encoder must be connected in any order to the three probes of the transistor tester 
before the start of the function. 
After starting the function the rotary knob must be turned not too fast.
If the test is finished successfully, the connection of the encoder switches  is shown symbolic in display row 2.
The tester finds out the common contact of the two switches and shows, if the indexed position has
both contacts in open state ('o') or in closed state ('C').
A rotary encoder with open switches at the indexed positions is shows in row 2 for two seconds as ''1-/-2-/-3 o''.
This type of encoder has the same count of indexed positions as count of pulses for every turn. 
Of course the pin number of the right common contact is shown in the middle instead of '2'.
If also the closed switches state is detected at the indexed positions, the row 2 of the display is also
shown as ''1---2---3 C'' for two seconds.
I don't know any rotary encoder, which have the switches always closed at any indexed position.
The interim state of the switches between the indexed positions is also shown in row 2 for a short time (\textless\(~0.5s\))
without the characters 'o' or 'C'.
If you will use the rotary encoder for handling the tester, you should set the Makefile option WITH\_ROTARY\_SWITCH=2
for encoders with only the open state ('o') and set the option WITH\_ROTARY\_SWITCH=1 for encoders 
with the open ('o') and closed ('C') state at the indexed positions.\\

\item[C(\(\mu F\))-correction]
With this menu function you can change a correction factor for bigger capacity values.
You can preset the same factor with the Makefile option C\_H\_KORR.
Values above zero reduce the output value of the capacity with this percent value, values below zero will increase
the shown capacity value. A short key press will reduce the correction value about 0.1\%, a longer key press will
increase the correction value. A very long key press will save the correction value.
It is a characteristic of the used measurement method, that capacitors with low quality like electrolytic type will
result to a too high capacity value. You can detect a capacitor with low quality by a higher value of the Vloss parameter.
High quality capacitors have no Vloss or only 0.1\%.
For adjusting this parameter you should only use capacitors with high quality and a capacity value above \(50\mu F\).
By the way the exactly capacity value of electrolytic capacitors is unimportant, because the capacity value differ
with temperature and DC-voltage.

\item[Selftest]
With the menu function ''Selftest'' a full selftest with calibration is done.
With that call all the test functions T1 to T7 (if not inhibited with the NO\_TEST\_T1\_T7 option) 
and also the calibration with external capacitor is done every time.\\

\item[Voltage]
The additional function ''Voltage'' (Voltage measurement) is only possible, if the serial output is deselected
or the ATmega has at least 32 pins (PLCC) and one of the additional pins ADC6 or ADC7 is used for the measurement.
Because a 10:1 voltage divides is connected to PC3 (or ADC6/7), the maximum external voltage can be \(50V\).
A installed DC-DC converter for zener diode measurement can be switched on by pressing the key.
Thus connected zener diodes can be measured also.
By selecting the POWER\_OFF option in the Makefile and without key pressing, the period of voltage measurement is limited to 4~minutes.
The measurement can also be finished with a extra long key press (\textgreater~4~seconds).

\item[Contrast] 
This function is available for display controllers, which can adjust the contrast level with software.
The value can be decreased by a very short key press or left turn with the rotary encoder.
A longer key press (\textgreater~0.4s) or a right turn of the rotary encoder will increase the value.
The function will be finished and the selected value will be saved nonvolatile in the EEprom memory 
by a very long key press (\textgreater~1.3s).

 \item[Show data]
The function ,,Show Data'' shows besides the version number of the software the data of the calibration.
These are the zero resistance (R0) of the pin combination 1:3, 2:3 and 1:2 .
In addition the resistance of the port outputs to the \(5V\) side (RiHi) and
to the \(0V\) side (RiLo) are shown.
The zero capacity values (C0) are also shown with all pin combinations (1:3, 2:3, 1:2 and 3:1, 3:2 2:1).
Next the correction values for the comparator (REF\_C) and for the reference voltage (REF\_R) are also shown.
With graphical displays the used icons for parts and the font set is also shown.
Every page is shown for 15 seconds, but you can select the next page by a key press or a right turn of the rotary encoder.
With a left turn of the rotary encoder you can repeat the output of the last page or return to the previous page.


\item[Switch off]
With the additional function ''Switch off'' the tester can be switched off immediately.\\

\item[Transistor]
Of course you can also select the function ''Transistor'' (Transistor tester) to return to a normal Transistor tester measurement. 
\end{description}

With the selected POWER\_OFF option in the Makefile, all additional functions are limited in time without interaction to prevent a discharged battery.


\section{Selftest and Calibration}

If the software is configured with the selftest function, the selftest can be prepared by connecting all three
test ports together and pushing of the start button.
To begin the self test, the start butten must be pressed again within 2 seconds, or else the tester will continue
with a normal measurement.

If the self test is started, all of the documented tests in the Selftest chapter \ref{sec:selftest} will be done.
If the tester is configured with the menu function (option WITH\_MENU), 
the full selftest with the tests T1 to T7 are only done with the ''Selftest'' function, 
which is selectable as menu function.
In addition the calibration with the external capacitor is done with every call from function menu,
otherwise this part of calibration is only done first time.
Thus the calibration with the automatically started selftest (shorted probes) can be done faster.
The repetition of the tests T1 to T7 can be avoided, if the start button is hold pressed.
So you can skip uninteresting tests fast and you can watch interresting tests by releasing the start button.
The test 4 will finish only automatically if you separate the test ports (release connection).

If the function AUTO\_CAL is selected in the Makefile, 
the zero offset for the capacity measurement will be calibrated with the selftest.
It is important for the calibration task, that the connection between the three test ports is relased 
during test number 4. 
You should not touch to any of the test ports or connected cables when calibration (after test 6) is done.
But the equipment should be the same, which is used for further measurements.
Otherwise the zero offset for capacity measurement is not detected correctly.
The resistance values of port outputs are determined at the beginning of every measurement with this option.\\

If you have selected the samplingADC function in the Makefile with the option ''WITH\_SamplingADC = 1'',
two special steps are included to the calibration precedure.
After the normal measuring of the zero capacity values, also the zero capacity values for the samplingADC function is
measured (C0samp). As last step of the calibration the connection of a test capacitor at pin~1 and pin~3 is requested for
later measurement of little coils with the message \mbox{1 \electricC 3  10-30nF[L]}.
The capacity value should be between \(10nF\) and \(30nF\),
to get a measurable resonant frequency by later parallel connection to a coil with less than \(2mH\).
For inspection of coils with more than \(2mH\) inductance the normal measurement method should result to a
sufficient accuracy. The parallel connection of the capacitor to use the other measurement method should
not be usefull.\\

A capacitor with any capacity between \(100nF\) and \(20\mu F\) connected to pin~1 and pin~3 is
required after the measurement of the zero capacity values.
To indicate that, the message \mbox{1 \electricC 3 \textgreater 100nF} is shown in row 1 of the display.
You should connect the capacitor not before the message C0= or this text is shown.
With this capacitor the offset voltage of the analog comparator will be compensated for better measurement
of capacity values.  
Additionally the gain for ADC measurements using the internal reference voltage will be adjusted too 
with the same capacitor for better resistor measurement results with the AUTOSCALE\_ADC option.
If the menu option is selected for the tester and the selftest is not started as menu function,
the calibration with the external capacitor is only done for the first time calibration.
The calibration  with the external capacitor can be repeated with a selftest call as menu selection.

The zero offset for the ESR measurement will be preset with the option ESR\_ZERO in the Makefile.
With every self test the ESR zero values for all three pin combinations are determined.
The solution for the ESR measurement is also used to get the values of resistors below \(10\Omega\) with
a resolution of \(0.01\Omega\).


\section{special using hints}
Normally the Tester shows the battery voltage with every start. If the voltage fall below a limit,
a warning is shown behind the battery voltage. If you use a rechargeable \(9V\) battery, you should replace
the battery as soon as possible or you should recharge.
If you use a tester with attached \(2.5V\) precision reference, the measured supply voltage will be shown
in display row two for 1 second with ''VCC=x.xxV''.

It can not repeat often enough, that capacitors should be discharged before measuring.
Otherwise the Tester can be damaged before the start button is pressed.
If you try to measure components in assembled condition, the equipment should be allways disconnected from power source.
Furthermore you should be sure, that no residual voltage reside in the equipment.
Every electronical equipment has capacitors inside!

If you try to measure little resistor values, you should keep the resistance of plug connectors and cables in mind.
The quality and condition of plug connectors are important, also the resistance of cables used for measurement.
The same is in force for the ESR measurement of capacitors.
With poor connection cable a ESR value of \(0.02\Omega\) can grow to \(0.61\Omega\).
If possible, you should connect the cables with the test clips steady to the tester (soldered).
Then you must not recalibrate the tester for measuring of capacitors with low capacity values,
if you measure with or without the plugged test cables. 
For the calibration of the zero resistance there is normaly a difference, if you connect the
three pins together directly at a socket or if you connect together the test clips at the end of cables.
Only in the last case the resistance of cables and clips is included in the calibration.
If you are in doubt, you should calibrate your tester with jumpers directly at the socket 
and then measure the resistance of shorten clips with the tester.

You should not expect very good accuracy of measurement results, especially the ESR measurement and the results of inductance measurement are not very exact.
You can find the results of my test series in chapter \ref{sec:measurement} at page~\pageref{sec:measurement}.

\section{Compoments with problems}
You should keep in mind by interpreting the measurement results, that the circuit of the TransistorTester is
designed for small signal semiconductors. In normal measurement condition the measurement current can only reach about \(6mA\).
Power semiconductors often make trouble by reason of residual current with the identification an the measurement of junction capacity value.
The Tester often can not deliver enough ignition current or holding current for power Thyristors or Triacs.
So a Thyristor can be detected as NPN transistor or diode. Also it is possible, that a Thyristor or Triac is detected as unknown.

Another problem is the identification of semiconductors with integrated resistors.
So the base - emitter diode of a BU508D transistor can not be detected by reason of the parallel connected
internal \(42\Omega\) resistor.
Therefore the transistor function can not be tested also.
Problem with detection is also given with power Darlington transistors. We can find often internal
base - emitter resistors, which make it difficult to identify the component with the undersized measurement current.

\section{Measurement of PNP and NPN transistors}
For normal measurement the three pins of the transistor will be connectet in any order to the measurement
inputs of the TransistorTester.
After pushing the start button, the Tester shows in row 1 the type (NPN or PNP), 
a possible integrated protecting diode of the Collector - Emitter path and the
sequence of pins. The diode symbol is shown with correct polarity.
Row 2 shows the current amplification factor \(B\) or \(hFE\) and the current, by which the
amplification factor is measured. If the common emitter circuit is used for the hFE determinatation,
the collector current \(Ic\) is output. If the common collector circuit is used for measuring of the
amplification factor, the emitter current \(Ie\) is shown.
Further parameters are shown for displays with two lines in sequence, one after the the other in line 2.
For displays with more lines further parameters are shown directly until the last line is allready used.
When the last line is allready used before, the next parameter is shown also in the last line
after a time delay automatically or earlier after a key press.
If more parameters are present than allready shown, a + character is shown at the end of the last line.
The next shown parameter is anyway the Base - Emitter threshold voltage.
If any collector cutoff current is measurable, the collector current without base current \(I_CE0\)
and the collector current with base connected to the emitter \(I_CES\) is also shown.
If a protecting diode is mounted, the flux voltage \(Uf\) is also shown as last parameter.


With the common Emitter circuit the tester has only two alternative to select the base current:
\begin{enumerate}
\item The \(680\Omega\) resistor results to a base current of about \(6.1mA\). 
This is too high for low level transistors with high amplification factor, because the base is saturated.
Because the collector current is also measured with a \(680\Omega\) resistor, the collector current
can not reach the with the amplification factor higher value.
The software version of Markus F. has measured the Base - Emitter threshold voltage in this ciruit (Uf=...).\\
\item The \(470k\Omega\) resistor results to a base current of only \(9.2\mu A\) .
This is very low for a power transistor with low current amplification factor.
The software version of Markus F. has identified the current amplification factor with this circuit (hFE=...).\\
\end{enumerate}

The software of the Tester figure out the current amplification factor additionally with the common Collector circuit.
The higher value of both measurement methodes is reported.
The common collector circuit has the advantage, that the base current is reduced by negative current feedback corresponding
to the amplification factor. 
In most cases a better measurement current can be reached with this methode for power transistors
with the \(680\Omega\) resistor and for Darlington Transistors with \(470k\Omega\) resistor.
The reported Base - Emitter threshold voltage Uf is now measured with the same current used 
for determination of the current amplification factor.
However, if you want to know the Base - Emitter threshold voltage with a measurement current of about \(6mA\),
you have to disconnect the Collector and to start a new measurement.
With this connection, the Base - Emitter threshold voltage at \(6mA\) is reported. The capacity value
in reverse direction of the diode is also reported.
Of course you can also analyse the base - collector diode.

With Germanium transistors often a Collector cutoff current \(I_{CE0}\) with currentless base or 
a Collector residual current \(I_{CES}\) with base hold to the emitter level is measured.
Only for ATmega328 processors the Collector cutoff current is shown in this case at the row 2 of the LCD 
for 5 seconds or until the next keypress before showing the current amplification factor. 
With cooling the cutoff current can be reduced significant for Germanium transistors.


\section{Measurement of JFET and D-MOS transistors}
Because the structure of JFET type is symmetrical, the Source and Drain of this transistores can not
be differed.
Normally one of the parameter of this transistor is the current of the transistor with the Gate at the same level as Source.
This current is often higher than the current, which can be reached with the measurement circuit of the TransistorTester
with the \(680\Omega\) resistor.
For this reason the \(680\Omega\) resistor is connected to the Source. Thus the Gate get with the growing of current a negative
bias voltage.
The Tester reports the Source current of this circuit and additionally the bias voltage of the Gate.
So various models can be differed.
The D-MOS transistors (depletion type) are measured with the same methode.

\section{Measurement of E-MOS transistors}
You should know for enhancement MOS transistors (P-E-MOS or N-E-MOS), that the measurement of the gate threshold voltage (Vth)
is more difficult with little gate capacity values. You can get a better voltage value, if you connect a capacitor with a value
of some nF parallel to the gate /source.
The gate threshold voltage will be find out with a drain current of about \(3.5mA\) for a P-E-MOS and about \(4mA\) for a N-E-MOS.
The RDS or better R\textsubscript{DSon} is measured with a gate - source voltage of nearly \(5V\), which is probably not the lowest value.

\section{Measurement of capacitors}
The capacity values are always computed from the time constant, which is build by the serial connection of the
build in resistors with the capacitor during charging. With little capacity values the \(470k\Omega\) resistors are
used for the measurement the time to reach a threshold voltage.
For bigger capacity values with some \(10\mu F\) the voltage grow at the capacitor is monitored after charge pulses with the
\(680\Omega\) resistors. With this voltage grow the capacity can be computed together with the count of fixed length pulses.
Very low capacity values can be measured with the samplingADC method.
For analysing the same load pulse is repeated many times and the voltage is monitored with the time shift of the ADC S\&H time
using interval-tics build from the processor clock. But a complete AD conversion take 1664 processor tics!
Up to 250 ADC samples are build by this way and from the voltage curve the capacity value is computed.
If the samplingADC function is selected in the Makefile, all capacitors with less than \(100pF\) are measured with
the samplingADC function in the capacitor-meter mode {\bf[C]}. The resolution is up to \(0.01pf\) with a clock 
frequency of \(16MHz\). The calibrated condition is difficult to build with this high resolution.
You can assume the use of the samplingADC methode every time, fractions of \(1pF\) are displayed at the screen.
By the way it should mentiored, that the junction capacitance of single diodes can also be measured with
this method. Because this method can measure the capacity value by charging or discharging, two capacity results are shown.
Both values differ be reason of the capacity diode effect.

\section{Measurement of coils}
The normal measurement of the inductance is based on the measurement of the time constant of the current grow.
The detection limit is about \(0.01mH\), if the resistance of the coil is below \(24\Omega\).
For bigger resistance values the resolution is only \(0.1mH\).
If the resistance is above \(2.1k\Omega\), this technique can never be used to detect coils.
The measurement results of this normal measurement is shown in the second line (resistance and inductance).
With the samplingADC method a resonant frequency of coils can be detected with greater inductance values.
If this effect is noticeable, the frequency and the quality factor Q of the coil is shown additionally in line 3. 

The method of resonant frequency measurement can also be used for the determination of the inductance value,
if a sufficient big capacitor mith know capacity value is connected parallel to a little inductance (\textless\(2mH\)).
With a parallel connected capacitor the normal measurement of inductance can no more operate well.
If the resonant frequency let assume a parallel connected capacitor, the inductance of the normal measurement
is not shown and the resistance value is shown in line 1.
For this resonant circuit the quality factor Q is also computed and shown behind the frequency value in line 3.
You can identify this type of measurement with the inductance value at the first position of line 2,
followed by the text '' if '' and the value of the assumed parallel capacity.
The value of this parallel capacitor can currently only be set with the calibration function (\mbox{1 \electricC 3 \(10-30nF\)(L)}).

For displays with only two lines, the content for the third line is shown time-delayed in line 2.
